\section*{Introduction}

Industrial anomaly detection plays an important role for ensuring quality control and minimizing defects in manufacturing processes \cite{chandola2009anomaly, iturbe2017towards}. Early detection of anomalies, often caused by equipment malfunctions or material defects, is crucial for mitigating financial losses and operational disruptions. Traditional methods for detecting these anomalies relied heavily on filtering and handcrafted features to distinguish between normal and abnormal samples \cite{chandola2009anomaly}. However, these approaches often perform poorly with complex and high-dimensional data, such as industrial images.

With the advent of deep learning, anomaly detection techniques have evolved significantly \cite{tao2022deep, cui2023survey}. In particular, unsupervised methods have gained attention due to the challenges and costs associated with obtaining labeled data for abnormal instances \cite{cui2023survey}. These methods typically train models using only normal samples, enabling the system to identify deviations from normality in unseen test samples that may contain both normal and abnormal data. Deep learning-based anomaly detection techniques primarily focus on two categories: reconstruction-based and feature-embedding methods. Reconstruction-based methods aim to restore anomalous regions by comparing reconstructed images with the originals, thus highlighting discrepancies at defect locations \cite{akcay2019ganomaly, yu2023unsupervised, liu2020towards}. Despite their success in detecting anomalies, these methods often struggle with the extraction of high-level semantic features, which limits their classification performance. On the other hand, feature-embedding methods leverage pre-trained networks to learn representations of normal samples \cite{bergmann2020uninformed, salehi2021multiresolution, wang2021student, cao2022informative}. Deviations from these learned representations signal the presence of anomalies. These methods often adopt a teacher-student architecture, where a backbone model serves as a teacher to train a student model for feature extraction. Anomaly detection is then performed by comparing feature maps from both models, with significant deviations indicating anomalies.

While the above techniques have proven effective, they often rely on RGB image data, which lacks the geometric context necessary for precise anomaly localization in industrial settings. Recent advancements in computer vision have explored the use of depth images and 3D point clouds to enhance detection by capturing geometric cues alongside traditional RGB features \cite{bergmann2023anomaly, rudolph2023asymmetric, horwitz2023back, wang2023multimodal}. However, despite the potential of these multimodal approaches, several limitations remain. First, the high-dimensionality of 3D point cloud data, coupled with the irregularity of point clouds, presents a significant challenge for traditional convolutional neural networks (CNNs), which are primarily designed for structured data. Another limitation lies in the computational complexity and resource demands of multimodal methods. Processing high-resolution depth maps or point clouds alongside RGB images requires substantial memory and processing power, which can make real-time industrial applications impractical. Memory-based methods, for example, store representative features from both modalities, but their reliance on large feature banks increases memory requirements and prolongs inference times, hindering their deployment in resource-constrained environments.

In this paper,  we propose an unsupervised anomaly detection framework that effectively leverages both visual and geometric information. Our approach begins by using pre-trained transformer-based models to separately extract 2D visual features from color images and 3D geometric features from point clouds. We then design a Visual-to-Geometric Feature Reconstruction network that takes the visual features as input and predicts the corresponding geometric features. Through the use of non-local attention mechanisms, the model captures global relationships between visual and geometric features, ensuring that long-range dependencies in the data are retained, while graph convolutional networks (GCNs) enable it to refine local geometric details by modeling high-order spatial relationships between neighboring visual tokens. During training, the network minimizes the difference between the predicted geometric features and the original geometric features extracted from the point cloud, which enables it to learn the correlation between appearance and geometry in normal objects. By focusing on reconstructing the geometric features based on visual inputs, the network captures how normal visual features translate into geometric representations, making it highly responsive to any anomalies in the test data. Since the training data contains only normal samples, the network learns the expected patterns and correlations between 2D and 3D features for normal objects. During inference, any significant deviation between the predicted and actual geometric features signals the presence of an anomaly.

In brief, the main contributions of our work are:

\begin{itemize}
    \item \textbf{Innovative Anomaly Detection Network}: We propose a novel unsupervised anomaly detection framework that reconstructs 3D geometric features from 2D visual features using a geometric feature prediction network, rather than directly fusing the modalities.
     
    \item \textbf{Visual-to-Geometric Feature Reconstruction}: The core innovation of our framework lies in the introduction of a visual-to-geometric feature reconstruction network, which predicts 3D geometric features from 2D visual inputs. This approach leverages two key techniques: (1) non-local attention mechanisms and (2) graph convolution networks (GCNs) to effectively capture both global and local relationships between 2D and 3D features.
     
    \item \textbf{Training Strategy}: We introduce a training strategy to minimize the difference between the predicted geometric features and the target geometric features extracted from point clouds, allowing the network to learn the normal correlations between 2D and 3D features.
    
    \item \textbf{State-of-the-Art Performance}: Our method demonstrates state-of-the-art performance on industrial anomaly detection tasks, with significant improvements in both detection accuracy and inference speed.
\end{itemize}
